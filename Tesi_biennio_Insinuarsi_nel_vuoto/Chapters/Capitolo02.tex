% !TEX TS-program = pdflatex
% !TEX encoding = UTF-8 Unicode

%************************************************
\chapter{Strumento di nuova concezione}
\label{Strumento di nuova concezione}
%************************************************

Come non cadere nel fascino della creazione tramite "tavole musicali" come avviene in Cage o evitare di intraprendere un percorso musicale legato al teatro, dove diventa più importante il gesto stesso che la composizione e la forma? Semplice, tramite un'analisi specifica degli strumenti in questione. \\
Appena la nostra ricerca, da ammaliante, diventa analitica, allora il tutto diventa più chiaro e si perdono anche le facili sperimentazioni sulla realtà e si va a conoscere una bibliografia sempre più estesa e ci si perde, verso, appunto, l'ignoto. \\
Reputo essenziale nella vita di un compositore di musica elettroacustica, l'avvicinamento al mondo \textit{liuteristico} e di costruzione, semplicemente perché il realizzare qualcosa di fisico, comporta il contatto con la materia e la possibilità di avvicinarsi \textit{con mano} al suono che si vuole andare a formare, o modificare e trasformare. \\
La scuola del compositore che immagina, scrive, e solo in seguito dà vita alle proprie opere, a mio parere, è ormai lontana, perché sovviene un nuovo importante aspetto legato al suono: la complessità spettrale legata alle armoniche, alla fusione e l’evoluzione di formanti, ma soprattutto alla caduta di un regime tonale e di armonia classica, verso \textit{micro-variazioni} e \textit{micro-ritmiche} legate sia all’universo microscopico che a quello macroscopico. \\
La fisica è giunta ormai a studiare delle parti sempre più piccole della materia, poiché proprio all’interno del più piccolo frammento di essa si può trovare la nostra origine e delle risposte legate all’evoluzione. 



%************************************************


\section{Unamolla}

Indico \textbf{Unamolla} come \textit{strumento di nuova concezione}, uno strumento composto da "una molla" in ferro armonico, una base in legno e un sistema di amplificazione a contatto. \\
Paolo Zavagna, in alcuni suoi appunti pubblicati sul suo sito internet\footnote{http://www.zavagna.it/paolo/}, redige una classificazione degli \textit{strumenti musicali elettronici}, qui definisce gli strumenti \textbf{Elettromagnetici}:

\begin{quotation}
Gli strumenti musicali elettromeccanici che producono il suono tramite un’azione meccanica di riproduzione di una forma d’onda, che può essere "scritta" su un nastro in movimento o su un disco rotante.\end{quotation}

Unamolla è quindi, uno strumento elettromagnetico di nuova concezione. \\

\subsection{Passaggio da installazione a strumento}

Nel 2018, quando ideai Sp.I.R.E. [installazione di molle e piastre regolate ed elettrificate], notai subito che un difetto assai visibile era la trasportabilità. \\
Se in passato bastava scrivere una partitura e nell’eventualità trasportare il mio personal computer, ora avevo creato un “oggetto” costruito appositamente per tale scopo e (per quanto ci fosse un progetto alle spalle) molto cose erano riproduzioni fisiche di schizzi creati con carta e penna. 
In seguito, tramite le mie ricerche sulla performance musicale e lo sviluppo di varie tecniche per la gestualità sullo Sp.I.R.E., decisi di utilizzare un oggetto sonoro per volta, dedicandomi ad un limitato numero di parti: due. Utilizzai quindi \textit{Unamolla} in acciaio ed una piastra in rame\footnote{le prime erano in ferro armonico e zincato, le seconde in acciaio}; infine, decisi di dedicarmi ad un unica parte e limitai la mia attenzione esclusivamente alla molla. \\
Se Sp.I.R.E. era uno strumento composto da 4 molle e 2 piastre in grado, tramite degli attuatori, di auto-eccitarsi, Unamolla è uno strumento meno complesso nella costruzione, ma più complesso nel sistema di ripresa. Lo Sp.I.R.E. aveva come sistema di ripresa solo 4 microfoni a contatto e due microfoni posizionati ai lati per donare una stereofonia all'ascolto. Nel caso di Unamolla, invece, la diffusione è \textbf{mono} ed è più complessa dello Sp.I.R.E., perché è in realtà una somma di più fonti di ripresa sonora collocate sullo strumento.

\subsection{Sistema di amplificazione \textit{UNAMOLLA}}

Durante la costruzione dello Sp.I.R.E., avevo notato che il segnale non era trasdotto perfettamente (utilizzavo solo dei piezoelettrici), mancava di attacco e le dinamiche erano davvero piatte. Aggiunsi quindi su una tavola di legno, prima uno, poi due humbucker (per chitarra e per basso) in aggiunta al piezoelettrico e fui soddisfatto del risultato. Ho potuto, quindi, definire \textit{Unamolla} come \textit{strumento musicale} ed iniziare la mia analisi gestuale.
Unamolla si può definire uno strumento a \textbf{spettro inarmonico} costituito, nel primo prototipo, da una lastra di ottone e di una molla in ferro-armonico, connesse da una tavola di legno di noce. Questo primo prototipo, che ho chiamato: Unamolla 1.0, ha subìto con il tempo delle variazioni ulteriori dalla figura che segue e, con il tempo ho diminuito ulteriormente la sua grandezza e ho diviso piastra e molla. \\
In figura 2, lo schema di costruzione di \textit{Unamolla 1.0}.

\begin{figure}

\begin{center}

\includegraphics[width=1.\textwidth]{unamolla_schema.png}

\caption{Schema Unamolla 1.0}

\label{fig:02_unamolla_01}

\end{center}

\end{figure}


Nello schema è disegnata la tavola di legno sulla quale è fissata una molla da 1,5 pollici e la piastra di rame, opportunamente disaccoppiata tramite una vite inferiore al contatto con il legno e una vite superiore di fissaggio. Sul lato sinistro della molla ho posto un doppio Humbucker per basso elettrico ed in seguito, ho diviso, alle due estremità, i due humbucker per chitarra. Al di sotto della parte in legno troviamo un piezoelettrico. \\
Unamolla 1.0 è stato il primo stadio, il primo prototipo di strumento. Anche se le dimensioni erano estremamente diminuite, la trasportabilità dello strumento era ancora in fase di cambiamento, dato che ogni volta dovevo smontarlo e montarlo, ma soprattutto alcuni gesti erano difficilmente eseguibili. \\
Decisi quindi di lavorare su una seconda \textit{Unamolla}  e ho creato il secondo prototipo: ho eliminato la lastra (come scritto in precedenza) per dare spazio ad un design più consono per eseguire determinati gesti legati anche all’uso dell’archetto. 

%************************************************


\section{Analisi acustica}

\begin{figure}

\begin{center}

\includegraphics[width=1.\textwidth]{unamolla_schema.jpg}

\caption{Schema Unamolla 1.2}

\label{fig:03_unamolla_02}

\end{center}

\end{figure}


L’analisi acustica di uno strumento amplificato di nuova concezione come \textbf{Unamolla} ha molteplici gradi di variabilità legati soprattutto alla gestualità dell’esecutore ed alla molteplicità di modalità eccitative della molla. \\
Non disponendo di una prassi esecutiva consolidata, è importante codificare in modo preciso e, quanto più possibile, riproducibile, i parametri che determinano il gesto e la modalità di eccitazione dello strumento. \\
Per iniziare un approccio di analisi ad Unamolla, ho deciso di utilizzare un archetto barocco per violoncello e restringere il range di analisi a determinati gesti. \\
\\
Come si nota in figura 3, ci possiamo avvalere di vari parametri per l’analisi di un'arcata, applicata direttamente sulla parte tratteggiata a destra.  \\

\subsection{Tecniche di ripresa}

Tutte le riprese sono state fatte con una scheda audio e sono a 48 kHz e 32bit. \\
I campioni sono tutti mono e il segnale che otteniamo è la risultante della somma (tramite saldatura) di:
\begin{itemize}

\item{Doppio Humbucker per basso}
\item{Doppio Humbucker per chitarra}
\item{Piezoelettrico}

\end{itemize}
Di seguito diamo le definizioni dei parametri gestuali presi in considerazione con i relativi range di valore:

\begin{itemize}
\item{\textbf{Archetto}: come mostrato in figura, consideriamo le tre zone punta, centrale, coda.}
\item{\textbf{Angolazione archetto}: l’angolazione presa in esame è quella dai 30 ai 60-70 gradi centigradi.}
\item{\textbf{Velocità}: lento, veloce, moderato.}
\item{\textbf{Tensione}: molla libera oppure fermata con la sola pressione di uno o due dita della mano libera.}
\item{\textbf{Pressione}: da 0 a 10, considerando lo zero con l’archetto semplicemente appoggiato, e il 10 come uno \textit{sf}.}
\end{itemize}

Prendiamo in esame, un determinato gesto che ha queste caratteristiche:
\begin{itemize}
\item{\textbf{Zona dell’archetto}: coda}
\item{\textbf{Angolatura}: 60 gradi}
\item{\textbf{Velocità}: lento}
\item{\textbf{Tensione}: molla fermata che tocca la base in legno a 1/3 a destra.}
\item{\textbf{Pressione}: 2}
\end{itemize}

In figura 4, lo spettro in frequenza del gesto. \\
\begin{figure}

\begin{center}

\includegraphics[width=1.\textwidth]{unamolla_analisi_02.png}

\caption{Unamolla, analisi spettro}

\label{fig:04_spettro_01}

\end{center}

\end{figure}

Nell’esecuzione di ogni gesto emergono degli aspetti legati all’interazione dell’esecutore con lo strumento e al feedback tattile alimentato dalla reazione dello strumento stesso all’intervento dell’esecutore. In questo caso la reazione più evidente è che tutto l’avambraccio e la mano avvertono una tenue modulazione sulle basse frequenze dovuta alla vibrazione della molla sulla base di legno. Questa vibrazione rimane fino ad una velocità moderata, con una pressione a 0.4-0.3 (in una scala da 0 a 1) e rimane fissa per tutte le zone dell’archetto, ma va a perdersi quando la pressione e la velocità aumentano, dato che entra in gioco un nuovo fattore che è legato sia alla vocalità dell’archetto, sia della molla: la molla inizia a vibrare di meno sulle basse frequenze e lo strofinio dell’arco si intensifica fino a suonare da sé.  \\
La bassa frequenza cerchiata è quella che crea quella rugosità del suono che possiamo apprezzare soprattutto a molla ferma con il dito e nella zona dell’archetto vicino alla coda verso il centro. Inoltre, all’interno dello spettro sottostante si possono identificare le punte degli inviluppi formantici. Uno a 130 hz e l’altro a 750. Durante l’esecuzione di questa parte, ho notato una vocalità ben precisa dello strumento, a circa 130 hertz (appunto) che come vediamo nello spettro successivo, si identifica in una parziale di una curva formantica. Facendo dei conti, ho notato che se si considerano determinati bin che compongono la curva formantica, si può notare come il MCD è intorno a 10-18 hz, nella figura notiamo un movimento non in banda audio a circa 18 hz. La parte nel cerchio rosso è sui 36 hz, la prima delle parziali che compongono uno spettro che di base è inarmonico, ma tramite determinati accorgimenti si può accentuare tale sua vocalità e rendere possibile la costruzione di determinati gesti ai fini del nostro volere compositivo. Riuscire anche a muoverci (o forse solamente a muoverci) in un universo micro-tonale. \\
Un altro fattore molto importante, notato durante l’esecuzione del gesto, è l’alta variabilità di suono che avviene anche con il cambio dell’angolatura dell’arco: a 45 e 60 gradi i crini dell’archetto non si incastrano con le spire della molla. Ovviamente può essere compositivamente interessante anche sfregare i crini all’interno delle spire della molla, ma in quel caso abbiamo bisogno di un archetto con poca pece, perché in quel caso la pece renderebbe difficile il movimento dell’archetto facendo incastrare le spire con le sezioni di crini che si vanno a formare.  \\
Quando muoviamo l’archetto sulla molla, ci scontriamo con la pressione, che va dosata, dato che ogni parte dell’archetto per via della tensione, suona in modo differente, quindi a seconda della frizione e della velocità, va mosso l’archetto in modo adeguato per creare un suono che risulti continuativo, che per gli scopi compositivi personali è molto importante. \\
Venendo da una composizione come L’albe nei varchi, che precede il solo con la molla in questo trittico, ho decisamente bisogno di un suono che sia in fase iniziale continuativo e soggetto a micromovimenti che possano rendere il movimento continuo non ripetitivo e che abbia realmente una direzione compositiva adeguata al mio stile. 
\\ \\
\textbf{Come si nota dalla seguente analisi di spettro-frequenza, dell’attacco dell’archetto sullo strumento, ci sono due curve formantiche intorno a 100 hz e a 700.} 

\subsection{Considerazioni sull’amplificazione}

Ho preferito limitare il campo di ripresa agli elettromagneti e ai microfoni a contatto, perché sono inerenti all’elaborazione del Live Electronics che andrò ad effettuare e soprattutto sono gli unici tre canali che amplifico. Ho deciso di unire tutte le riprese audio in un’unica uscita fisica. \\
Non ho notato cancellazioni fisiche del segnale. L’unione delle fonti sonore è il risultato di varie prove effettuate per avere più dinamica ma soprattutto più presenza dello strumento e uno spettro più ricco. La somma dei segnali è poi il reale suono che ho ricercato per la molla. \\
In futuro, come esplico nello schema in figura 5, cercherò di creare una cassa di risonanza \textbf{a lunghezza variabile} con la quale poter enfatizzare e modificare l’amplificazione del mio strumento e non delimitare più il campo di amplificazione al contatto, ma estenderlo ad una microfonazione differente (tramite la cassa di risonanza ed un microfono a condensatore).
\begin{figure}

\begin{center}

\includegraphics[width=1.\textwidth]{unamolla_2_schema.png}

\caption{Unamolla 2.0, design di Unamolla con cassa di risonanza variabile}

\label{fig:05_unamolla_2.0}

\end{center}

\end{figure}



\subsection{Conclusioni}

Le prime domande che mi sono poste era riguardo la comparazione con strumenti reali e la possibilità di farli convivere all’interno di una partitura. \\
Per quanto riguarda lo Sp.I.R.E. ho intrapreso uno studio gestuale molto legato al mondo percussivo. Nel caso di Unamolla, ho deciso di avvicinarmi più al mondo degli archi e l’analisi gestuale inizia con l’utilizzo di un archetto barocco per violoncello. I parametri in gioco sono diversi, l’idea di trovare una vocalità nello strumento sta appunto nella convivenza di vari fattori.\\
L’archetto è stato posizionato al margine sinistro della molla. La molla è stata “fermata” con un dito creando una sorta di nodo a 1/3  dal margine destro, subito dopo il primo humbucker per chitarra da destra. Abbiamo tre parametri che entrano in gioco, in base ai quali è stata fatta la classificazione del gesto sono i seguenti, legate a determinati parti della molla sulla quale avviene la speculazione.\\

\begin{itemize}
\item{\textit{Velocità}: dato che parliamo di uno strumento pluri-amplificato, si nota subito, al tocco, che tra lento e moderato può rientrare un universo di micro velocità che scaturiscono interessanti spunti di ricerca.}

\item{\textit{Tensione}: molla libera, fermata con la sola pressione di uno o due dita della mano libera dell’archetto. Questa seconda tipologia di tensione ha come risultato l’eliminazione delle \textit{auto-vibrazioni} indotte alla molla e la possibilità quasi di “intonarla”. Inoltre si può fermare la molla anche facendole toccare la base in legno sempre pigiando con un dito nella parte centrale; il risultato sarà una frequenza più alta della precedente e pari quasi al doppio della frequenza precedente: 130Hz. Questo comporta un indice di vibrazione sulle basse frequenze della molla, che diminuisce (o addirittura quasi scompare) quando viene fermata.}
\item{\textit{Zona dell’archetto}: Come per la velocità, queste parti contengono un mondo di microgesti che se uniti anche a velocità e modifiche della tensione, riescono a trasformare minimamente il suono inarmonico, riuscendo (anche tramite un’elaborazione) a eccitare la molla a tal punto da riuscire a sottolineare una vocalità intrinseca dello strumento.}
\item{\textit{Pressione}: ultimo ma non meno importante parametro, rende possibile il cambio timbrico della molla ed entrano in gioco parti inerenti all’archetto stesso e ad armoniche che possono venir fuori nel momento in cui si uniscono vari fattori, tra cui una determinata posizione dell’archetto, la velocità e la pressione. Ho adottato una scala da 0 a 1, dove 0.01 ad esempio è l’archetto in movimento semplicemente poggiato, e 1 è uno sf.
Bisogna avere molta cura nella pressione dell’arco, perché al minimo sbaglio suona l’arco stesso, quasi a parere un armonico, e dirige l’attenzione dell’ascoltatore verso frequenze molto acute, tralasciando le medio basse che perdono di intensità.}
\end{itemize}
\textbf{Tutti questi fattori possono essere riportati in partitura per precisare a quale suono si vuole puntare tramite un determinato gesto.} \\
\\
\textit{Nella difficoltà esecutiva di uno strumento con uno spettro complesso come quello della molla, possiamo sentire risuonare vari strumenti, come trombe, corde di contrabbasso e anche una minima vocalità se si vanno ad enfatizzare delle parti specifiche dello spettro.}



%************************************************

\section{Catalogazione degli strumenti, gesto-segno}

Per avvicinare al mondo degli strumenti musicali di liuteria classica, lo strumento Unamolla, ho deciso di intervallare la performance-solo dello strumento, con due composizioni: una per sassofono soprano tape ed elaborazione; un duo per strumento e sassofono tenore. Prima di iniziare la scrittura del duo, ho deciso di stabilire dei parametri sui quali lavorare, delle parti simili a livello spettrale tra i due strumenti. In primo luogo ho pensato ai multifonici e in seguito, nella parte del duo, alla quantità di rumore indotta nello strumento tramite un soffio più deciso nell'insufflazione  con cambiamenti della posizione del labbro inferiore. \\
La catalogazione che segue, è stata ideata per riuscire ad individuare gesti simili tra questi due strumenti di famiglie differenti: una appartenente all’universo temperato, l’altra, al contrario, appartenente all’universo della molla, in teoria più vicina a quello dei rumori. \\
Insieme ad un violista abbiamo riscontrato che ci sono varie parziali all’interno dello spettro della molla, come si può riscontrare anche nell’analisi. C’è una nota principale che è il LA, ma attorno, nello spettro della molla si possono definire sia un Si che un Sol diesis crescente che un La un’ottava superiore e un Sib. \\
Esistono varie aree di studio con le quali possiamo lavorare per avere un approccio reale all’unione di strumenti occidentali e strumenti di nuova creazione: \\
\begin{itemize}
\item{\textit{gestuale}: come detto in precedenza, studio legato al gesto fisico e successivamente alla creazione di un segno o simbolo che, ben spiegato in legenda, rende possibile l’esecuzione di una determinata cellula musicale;}
\item{\textit{comparativa}: strumenti reali in comparazione agli strumenti di nuova creazione;}
\item{\textit{compositivo}: unione ritmico-melodica su base formale e timbrica;}
\end{itemize}
A livello umanistico si stanno facendo ad oggi delle indagini riguardo il gesto. Non più una visione solo temporale dell’utilizzo di varie tecniche, ma anche spettrale e spaziale. Lo strumento d’analisi è, quindi, nel dominio congiunto di tempo e frequenza. Ovvero, tramite software specifici si dà spazio ad un’analisi che ha come base lo studio delle armoniche o delle accentuazioni a livello frequenziale in determinati gesti fatti sullo strumento. \\

\subsection{Strumenti a confronto: catalogazione}

Per fare un’analisi completa delle tecniche ho voluto stilare una serie di tecniche e gesti uguali per ogni strumento che vado ad analizzare. E’ ovvio che l’utilizzo delle chiavi è strettamente legato al flauto, ma a mio parere documentare tutte le tecniche rende ancora più palese la possibilità di rendere omogeneo questo studio.

\begin{table}[htp]
\caption{Sassofono Soprano, tabella comparativa tecniche}
\begin{center}
\begin{tabular}{|c|c|c|}
\hline
\textbf{Tecniche/Strumento} & \textbf{Unamolla} & \textbf{Sassofono Soprano} \\
\hline
Pizzicato & SI & NO \\
\hline
Grattato & SI & NO \\
\hline
Flautato & SI & SI \\
		&con archetto & \\
\hline
Nota lunga & SI & SI \\
		&con motore & \\
\hline
Nota corta & Mallet & SI \\
		&in metallo & \\
\hline
Trillo & NO & SI \\
\hline
Frullato & SI & SI \\
		&con motore & \\
\hline
Glissato & NO & SI \\
		& & ma solo su \\
				& & piccoli intervalli \\

\hline

Vibrato & NO & SI \\

\hline

Variazione & SI & SI \\
di quarto di tono & con motore & \\
& o archetto & \\

\hline
Variazione di tono & SI & SI \\
& solo se si crea & \\
& un nodo centrale & \\
\hline
Fruscìo & SI & SI \\
\hline
Soffio & NO & NO \\
\hline
Chiavi & NO & SI \\
\hline
Armonici & SI & SI \\
\hline
Bicordi & NO & MULTIFONICI \\
\hline
Tricordi & NO & MULTIFONICI \\
\hline
Accordi & NO &NO \\
\hline
\end{tabular}
\end{center}
\label{default}
\end{table}%

Come possiamo vedere dalla prima tabella, alcune tecniche che sono di natura come per uno strumento come il SASSOFONO SOPRANO, diventano un po’ più complessi da riprodurre per la molla. Ad esempio il frullato non può essere riprodotto, ma può venire eseguito dalla molla tramite un motorino continuo (vibratore o rasoio per capelli). 
Ho voluto anche specificare la presenza o meno dell’archetto con la molla per poter scrivere più precisamente anche i segni inerenti alla natura della tecnica eseguita sul nuovo strumento.
Ultimo gesti presente in Unamolla è legato al suo feedback. Tramite un overdrive della BOSS e un CRY-BABY (pedale wah-wah) ho la possibilità di eccitare determinate armoniche della molla e di riuscire, tramite il pedale, a controllare il feedback, riuscendo a definire dei piccoli cambiamenti microtonali.





