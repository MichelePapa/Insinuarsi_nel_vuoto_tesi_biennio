% !TEX TS-program = pdflatex
% !TEX encoding = UTF-8 Unicode

%************************************************
\chapter*{Bibliografia}
\label{chp:Bibliografia}
%************************************************
\textit{Libri e articoli citati nella tesi:} \\
\\
G.A.T.M. Bollettino, a cura di Lelio Camilleri \textit{Semestrale di analisi e teoria musicale}, Anno V, Numero 1, Monografie GATM 1998, edito dal gruppo di analisi e teoria musicale, Bologna, Luglio 1998 \\
\\
A.A.V.V. a cura di Agostino di Scipio e Paolo Zavagna \textit{Musica/Tecnologia, Rivista della fondazione Ezio Franceschini}, ISSN 1974-0042, Firenze University Press, 10.2016 \\
\\
A.A.V.V. a cura di Vincenzo Santarcangelo, \textit{Have your trip, La musica di Fausto Romitelli}, Auditorium edizioni, 2014 \\
\\
Hans Bellmer, \textit{Anatomia dell’immagine} Adelphi edizioni s.p.a. Milano, 2001 \\
\\
Luciano Berio, \textit{Intervista sulla musica}, Edizioni Laterza, Bari 2011 \\
\\
Walter Branchi, \textit{Tecnologia della musica elettronica} (con prefazione di Domenico Guaccero), Lerici, Roma, 1977 \\
\\
Sergio Cingolani e Renato Spagnolo, \textit{Acustica musicale e architettonica}, UTET, Torino, 2004 \\
\\
Macdonald Critchley, \textit{Il linguaggio del gesto}, “Il pensiero Scientifico” Editore, Roma, 1979 \\
\\
Francesco Galante Nicola Sani, \textit{Musica Espansa, Percorsi elettroacustici di fine millennio}, Casa Ricordi LIM Ediitrice, 2000 \\
\\
Wassily Kandisky, \textit{Punto, linea, superficie}, Adelphi Edizioni, Roma, 1968 \\
\\
Paolo Mauri, \textit{Buio}, Casa editrice Einaudi, Trento, aprile 2007 \\
\\
Netti-Weiss, \textit{The Techniques of Saxophone Playing. Die Spieltechnik des Saxophons} \\
\\
Luigi Nono, \textit{La nostalgia del futuro, Scritti scelti 1948-1986}, Gruppo editoriale Il Saggiatore s.p.a., Milano 2007 \\
\\
Tanja Orning \textit{Pression} (a performance study) Norwegian Academy of MusicMusic Performance Research Copyright © 2012 Royal Northern College of Music Vol. 5 \\
\\
Henri Pousseur, \textit{Musica, semantica, società}, traduzione di Eugenio Costa per Casa editrice Valentino Bompiani \& C. S.p.a., Milano, 1972 \\
\\
Curtis Roads, \textit{The Computer music tutorial}, 1996 \\
\\
Simone Santi Gubini, \textit{Difettosità timbrica e subarmoniche}, Masterclass Emufest  2017, Roma \\
\\
Leonardo Scopece, \textit{Riprese Olofoniche e Ambisoniche, Il sistema 3D-VMS}, Volume quinto de LeMiniSerie, iniziativa del Centro Ricerche e Innovazione Tecnologica della RAI, settembre 2011 \\
\\
Giacinto Scelsi, \textit{Il sogno 101}, Quodlibet S.r.l., Macerata 2010 \\
\\
R. Murray Schafer, \textit{The tuning of the world}, Alfred A. Knopf, New York, 1977 \\
\\
R. Murray Schafer, \textit{Il paesaggio sonoro}, Ricordi S.r.l. e LIM Editrice S.r.l., 1985 \\
\\
Alan Stones, \textit{The Analysis of Mixed Electroacoustic Music, Kaija Saariaho's Verblendungen, a case study}
\\
Iannis Xenakis, a cura di Agostino di Scipio, \textit{Universi del suono, Scritti e interventi 1955-1994}, Ricordi S.r.l. e LIM Editrice S.r.l., 2003 \\
\\
Trevor Wishart, \textit{On Sonic Art, a new and revised edition edited by Simon Emmerson}, Published in The Netherlands by Harwood Academic Publishers, Amsterdam, 1996 \\
\\
Enore Zaffiri, \textit{Due scuole di musica elettronica in Italia} Silva Editore, Milano, 1968 \\
\\
Laura Zattra, textit{ANALYSIS AND ANALYSES OF ELECTROACOUSTIC MUSIC}, University of Padua, Department of Visual Arts and Music, Volume 36, Sound and Music Computing (SMC05), Salerno, Italy, 2005