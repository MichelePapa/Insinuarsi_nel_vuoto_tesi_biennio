% !TEX TS-program = pdflatex
% !TEX encoding = UTF-8 Unicode

%************************************************
\chapter{Riflessioni e conclusioni}
\label{chp:Riflessioni e conclusioni}
%************************************************

Le mie conclusioni? \\
Penso che ci troviamo, anche per via della società, ad inseguire nella vita, un immagine di noi stessi, tentiamo di cercarci al di fuori, quando la nostra essenza è semplicemente lì. Davanti lo specchio, nelle nostre esperienze di vita, ma soprattutto in un metodo. \\
Saper agire e non reagire, a mio parere, è il primo passo. \\
Ogni giorno, qualunque cosa accada, si rimane sempre un io “pulsante”. Non inteso come oggetto o soggetto, ma come corpo vibrante in connessione con tutte le energie umane e non, che ha attorno. \\
Non sto dirigendo le mie conclusioni, né il mio discorso, verso pratiche zen o olistiche; mi riferisco in modo assoluto alla semplicità con la quale sono arrivato, dopo tutti questi anni, ad intraprendere, ora, il mio cammino. 
\\ \\
Ogni algoritmo al suo interno ha un algoritmo più semplice, fino ad arrivare alle operazioni basilari, ogni scomposizione ritmica, anche di quelle più complesse, ha al suo interno una ritmica più semplice e così via. Così la nostra percezione del suono e della forma di esso può andare fino al midollo dello spettro e della percezione che abbiamo di esso. \\
A mio parere si dicono sempre pi\'u parole di quelle che servono e non si lascia quasi mai far respirare l'ambiente circostante, che avrà sempre al suo interno qualcosa da raccontare. \\
La semplicità che genera complessità, la possibilità di confrontarsi con un vuoto, una mancanza continua, che appena si percepisce, si annuncia e non preannuncia: si materializza. Da lì, abbiamo a mio parere solo due possibilità: la prima è colmarlo con espedienti, ideali, note. La seconda è provare a inseguire quel vuoto incolmabile che crea l’imprevisto, l’indagine, il seguire una ricerca e intraprendere un cammino nel buio, demarcato solo dai nostri intenti. Insinuarsi nel vuoto, infine, sempre con un briciolo di umiltà e uno sguardo verso il futuro, con quella nostalgia perpetua che non è freno, ma moto di un’estetica che non si trancia o si ricrea, ma si trasforma e modella costantemente come fa il nostro essere in vita e così, la natura.