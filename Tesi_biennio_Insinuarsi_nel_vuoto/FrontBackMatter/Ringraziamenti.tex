%!TEX TS-program = pdflatex
%!TEX encoding = UTF-8 Unicode
%!TEX root = ../2018-03-26-papa-vitres-de-son.tex

%*******************************************************
% Introduzione
%*******************************************************

\chapter*{Ringraziamenti}
\pdfbookmark{Ringraziamenti}{Ringraziamenti}

\textit{I più vivi ringraziamenti a tutti i colleghi ed i maestri della} \textbf{Scuola di musica elettronica} \textit{del Conservatorio di Santa Cecilia che mi hanno supportato (e sopportato), durante la stesura della tesi e del lavoro di ricerca.} \\
\\
\textit{Ringrazio Danilo Perticaro, per la viva partecipazione al progetto di ricerca.}
\\ \\
\textit{Ringrazio tutti per la partecipazione e la presenza nella mia vita, come comparse o presenze e, in qualche caso, per fortuna, assenze.} \\ \\

textit{Ringrazio tutti, un viaggio lungo, pieno di occhi e suoni meravigliosi. Un traguardo che è solo un inizio, un sogno, che diventato realtà, nella complessità della sua vita, mi fa chiudere citando uno dei miei poeti più vicini, Giacomo Leopardi:} \\
\\
\subsection*{XII - L'Infinito}
\leftskip=1cm
	\textit{Sempre caro mi fu quest'ermo colle,\\
	e questa siepe, cha da tante parte \\
	dell'ultimo orizzonte il guardo esclude. \\
	Ma sedendo e mirando, interminati \\
	spazi di l\'a da quella, e sovrumani \\
	silenzi, e profondissima quiete \\
	io nel pensier mi fingo; ove per poco \\
	il cor non si spaura. E come il vento \\
	odo stormir tra queste piante, io quello \\
	infinito silenzio a questa voce \\
	vo comparando: e mi sovvien l'eterno, \\
	e le morte stagioni, e la presente \\
	e viva, e il suon di lei. Così tra questa \\
	immensit\'a s'annega il pensier mio: \\
	e il naufragar m'è dolce in questo mare.} \\

\clearpage

		    
