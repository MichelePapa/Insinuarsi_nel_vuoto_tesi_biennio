%!TEX TS-program = pdflatex
%!TEX encoding = UTF-8 Unicode
%!TEX root = ../2018-03-26-papa-vitres-de-son.tex

%*******************************************************
% Introduzione
%*******************************************************
\chapter*{Introduzione}

\pdfbookmark{Introduzione}{Introduzione}

\begin{flushright}
		\textit{Fatto sì è che qualsiasi creazioni mi ha sempre affascinato: vedere crescere le cose, essere create lì per lì sotto gli occhi... si tratti di una statua o di un quadro, o anche di un vestito nuovo, trovo che è sempre una nascita, una creazione che ha qualcosa di veramente emozionante. \footnote{Giacinto Scelsi, \textit{Il sogno 101}, Quodlibet S.r.l., Macerata 2010}}
	\end{flushright}

La musica elettronica è un mondo vasto, pieno di incantevoli afflati e meravigliose esplosioni. Un enorme calderone, ormai, di intenti, di sotterfugi e di bellezza, che tende a far trasparire a volte, i limiti dello strumento musicale o elettronico e allo stesso tempo le sue qualità. La tecnologia ci offre la possibilità di accedere a nuove tipologie di analisi e di elaborazione del segnale in tempo reale: questo deve essere un pregio del mondo contemporaneo, una spinta e non un riflesso. \\
Evitare la consequenzialità, la ridondanza, la ripetizione, è uno scopo che ho sempre cercato di raggiungere nelle mie composizioni. Riuscire ad avere una forma che mi desse la possibilità di far trasparire la continuità del tempo, come avviene al di fuori della composizione stessa e per quel postulato della fisica per il quale: 
\[ 
\textbf{"nulla si crea, nulla si distrugge, tutto si trasforma".}
\]
Continuamente il mio pensiero si dirige verso tale postulato. Considero la creazione musicale come un immagine mentale di un progetto, che rende possibile, nel reale poi, la trasformazione dei materiali grezzi in uno strumento: dall'informale al formale.
Lo strumento musicale nelle mani dell'esecutore o del performer, si tramuta in suono che dà vita alla composizione. Tale composizione indica un punto lontano da raggiungere che è la direzione musicale da noi scelta e non deve avere ambiguità esplicativa. \\ 
Questa tesi di ricerca è una sintesi di tali concetti: costruzione, analisi e composizione elettroacustica: formata da due movimenti, che prendono il nome di \textit{varchi}. I varchi temporali sono come i tempi che attraversiamo nella vita e, di seguito, ci lasciamo alle spalle, mantenendone la memoria. Ho tentato di insinuarmi in forme musicali strettamente connesse, all'interno di una timbrica rugosa ma nello stesso tempo lieve, alla maniera degli \textit{spettralisti}, dirigendo la mia attenzione ad un metodo legato ad attese di avvenimenti sempre differenti, ad affermazioni di gesti intensi, tramite un dialogo musicale che sfocia in un incastro così forte da diventare \textit{fusione}. \\
Spero che tutto ciò avvenga, in questi due varchi, che fanno il movimento dell'opera: \textbf{Insinuarsi nel vuoto}. 

\begin{small}
\begin{quotation}
\textit{Di ogni fenomeno si può fare esperienza in due modi. \\
Questi due modi non sono arbitrari ma connessi ai fenomeni - \\
essi vengono derivati dalla natura dei fenomeni, da due proprietà degli stessi: \\
esterno - interno\footnote{Wassily Kandisky, \textit{Punto, linea, superficie}, Adelphi Edizioni, Roma, 1968}}
\end{quotation}
\end{small}

Dopo la mia tesi compositiva del triennio \textit{Vitres de son, come un rosone nel cuore di un tempio immenso}, ho portato avanti la mia ricerca sui \textit{mollofoni} e sullo studio dello spettro inarmonico e la sua capacità di interagire con materiali sonori legati a strumenti di liuteria classica, come il brano \textit{Trapassato dal Futuro}\footnote{Trapassato dal futuro, composizione per Violoncello, Sp.I.R.E. e \textit{live electronics}, 2018, Prima assoluta: Istitut Goethe, Roma} scritto nel 2018 per il CRM (Centro di Ricerche musicali) di Roma, nella manifestazione \textit{ArteScienza 2018}. \\
La possibilità di interazione tra strumenti come violoncello o strumenti a fiato come il sassofono soprano\footnote{Presente nei due movimenti composti per la tesi}, è un ambito di ricerca che mi è molto a cuore da anni e che cerco di analizzare, facendo mio il concetto in ogni composizione. \\
L'ambito di ricerca è inerente allo spettro inarmonico della molla e alla sua capacità di divenire familiare ai suoni dell'ambiente cittadino nel quale siamo immersi. Finalmente eliminare gli ideali, per rendere familiare la composizione colta ad un ambiente, a tratti ostico, dove la direzione musicale prende un altro percorso: essere presente nella vita quotidiana e rendere intellettualmente più interessante il mondo circostante, dargli una forma per scrostargli quella caratteristica ansiogena e macchinosa che vuole dargli il capitalismo. Non fuggire da un ambiente comune che ormai abbiamo creato, ma dargli uno spiraglio per l'ispirazione e farlo diventare fonte di creazione. \\
Infine, ho focalizzato la mia creazione dello strumento ad una delle parti: la molla. \\
Finalmente la molla viene amplificata da un microfono a contatto e due magneti che confluiscono in un'unica fonte sonora, prendendo il nome di \textbf{Unamolla}.